\documentclass{beamer}

\usepackage{fancyvrb,tikz}

\usetikzlibrary{automata}

\usetheme{Dresden}

\mode<presentation>

\title[Tolkienizer]{Tolkienizer}
%\subtitle{}
%\author[]{}

\begin{document}

\begin{frame}
\titlepage
\end{frame}

\begin{frame}
   \frametitle{Contents}
   \tableofcontents[pausesections]
\end{frame}

\section{Introduction}

\begin{frame}
   \frametitle{Sean}
\end{frame}

\section{Who is this Markov, and why does he Hide his Models}

\subsection{Markov Chain}

\begin{frame}
   \frametitle{What is a Markov Process}
\end{frame}

\begin{frame}
   \frametitle{}
\end{frame}

\begin{frame}
   \frametitle{Stochastic Process}
\end{frame}

\subsection{Hidden Markov Models}
\begin{frame}
   \frametitle{}
\end{frame}

\begin{frame}
   \frametitle{Stochastic Process}
\end{frame}

\subsection{How we use them}
\begin{frame}
   \frametitle{How we use them}
\end{frame}

\section{Development Process}

\begin{frame}
   \frametitle{Overall Design}
   \begin{itemize}
      \item Two step process
      \begin{enumerate}
         \item Learn from a large set of words in a language
         \item Produce words that ``seem'' like they are in the language
      \end{enumerate}
   \end{itemize}
\end{frame}

\end{document}
