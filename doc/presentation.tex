\documentclass{beamer}

\usepackage{fancyvrb,tikz}

\usetikzlibrary{automata}

\usetheme{Dresden}

\mode<presentation>

\title[Tolkienizer]{Tolkienizer}
%\subtitle{}
%\author[]{}

\begin{document}

\begin{frame}
\titlepage
\end{frame}

\begin{frame}
   \frametitle{Contents}
   \tableofcontents[pausesections]
\end{frame}

\section{Introduction}

\begin{frame}
   \frametitle{What is Natural Language Processing?}
   \begin{itemize}
      \item Study of human language using computers
      \item Often uses statistical machine learning techniques
      \item Subfields include:
      \begin{itemize}
         \item Morphology
         \item Parsing
         \item Natural Language Understanding
      \begin{itemize}
   \end{itemize}
\end{frame}

\begin{frame}
   \frametitle{Generating words with Tolkienizer}
   \begin{itemize}
      \item Task: generate ``real sounding'' words for a language
      \item Similar to morphology - identifying key sounds instead of roots
      \item Example: create new ``Elvish sounding'' words from samples of Tolkien's Elvish
      \item Implementation: Hidden Markov Models
   \end{itemize}
\end{frame}

\section{Who is this Markov, and why does he Hide his Models}

\subsection{Markov Chain}

\begin{frame}
   \frametitle{What is a Markov Process}
\end{frame}

\begin{frame}
   \frametitle{}
\end{frame}

\begin{frame}
   \frametitle{Stochastic Process}
\end{frame}

\subsection{Hidden Markov Models}
\begin{frame}
   \frametitle{}
\end{frame}

\begin{frame}
   \frametitle{Stochastic Process}
\end{frame}

\subsection{How we use them}
\begin{frame}
   \frametitle{How we use them}
\end{frame}

\section{Development Process}

\begin{frame}
   \frametitle{Sam}
\end{frame}

\end{document}
